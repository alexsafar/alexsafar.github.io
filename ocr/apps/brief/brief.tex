\documentclass[10pt]{article}  

%%%%%%%% Preamble %%%%%%%%%%%%
\title{Project Brief}
\usepackage[english]{babel}
\usepackage[utf8]{inputenc} 
\usepackage{amsmath} 
\usepackage{amssymb} 
\usepackage{graphicx} 
\usepackage{color} 
\usepackage{subfigure} 
\usepackage{float} 
\usepackage{capt-of} 
\usepackage{sidecap} 
\sidecaptionvpos{figure}{c} 
\usepackage{caption} 
\usepackage{commath} 
\usepackage{graphicx, amsmath, amsthm, latexsym, amssymb, amsfonts, epsfig, float, enumerate, color, listings,  graphicx, fancyhdr}

\usepackage{cancel} 
\usepackage[table,xcdraw]{xcolor} 
\usepackage{anysize}

\marginsize{2cm}{2cm}{2cm}{2cm}

\usepackage{appendix}

\usepackage[colorlinks=true,plainpages=true,citecolor=blue,linkcolor=blue]{hyperref}
\usepackage{fancyhdr} 
\pagestyle{fancy}
\fancyhf{}

\fancyfoot[R]{\footnotesize Scott Morgan}
\fancyfoot[C]{\thepage}
\fancyfoot[L]{\footnotesize Creating Mobile Applications for Business (Unit 14)}
\renewcommand{\footrulewidth}{0.4pt}


\usepackage{listings}
\definecolor{dkgreen}{rgb}{0,0.6,0}
\definecolor{gray}{rgb}{0.5,0.5,0.5} 
\lstset{language=Matlab,
	keywords={break,case,catch,continue,else,elseif,end,for,function,
		global,if,otherwise,persistent,return,switch,try,while},
	basicstyle=\ttfamily,
	keywordstyle=\color{blue},
	commentstyle=\color{red},
	stringstyle=\color{dkgreen},
	numbers=left,
	numberstyle=\tiny\color{gray},
	stepnumber=1,
	numbersep=10pt,
	backgroundcolor=\color{white},
	tabsize=4,
	showspaces=false,
	showstringspaces=false}

\newcommand{\sen}{\operatorname{\sen}}

%%%%%%%% END PREAMBLE %%%%%%%%%%%%

\begin{document}
	%%%
	\begin{center}																		%%%
		\newcommand{\HRule}{\rule{\linewidth}{0.5mm}}									%%%\left
		%%%
		%\begin{minipage}{0.48\textwidth} \begin{flushleft}
				%\includegraphics[scale = 0.2]{cardiff.jpg}
		%\end{flushleft}\end{minipage}
		%\begin{minipage}{0.48\textwidth} 
			\begin{flushright}
				\includegraphics[scale = 1]{bcoll.png}
		\end{flushright}
	%\end{minipage}
		
		%%%
		\vspace*{3cm}								%%%
		%%%	
		\textsc{\huge Cambridge Technicals IT}\\[1.5cm]
		
		\vspace{3cm}											%%%
		%%%
		\HRule \\[0.4cm]																	%%%
		{ \huge \bfseries Raspberry Pi Robo Rally: Project Brief}\\[0.4cm]	%%%																	%%%
		\HRule \\[1.5cm]																	%%%
		%%%
		
		\vspace{2cm} 																				
		\begin{center}																					
			{\large \today}																	%%%
		\end{center}												  						
	\end{center}	
	\vspace*{3cm}																			
	
	\begin{minipage}{0.52\textwidth}													%%%
		\begin{flushleft} \large															%%%
			\emph{Contact: %(Cardiff University):
			}\\	
			Scott Morgan\\
			web: scott3142.com\\
			e-mail: MorganSN@cardiff.ac.uk\\
		\end{flushleft}																		%%%
	\end{minipage}
	\begin{minipage}{0.46\textwidth}	
		%%%
		\begin{flushright} \large															%%%
			Claire George\\
			e-mail: CGeorge@bridgend.ac.uk\\
		\end{flushright}																		%%%
	\end{minipage}
	
	
	\newpage																		
	%%%%%%%%%%%%%%%%%%%% TERMINA PORTADA %%%%%%%%%%%%%%%%%%%%%%%%%%%%%%%%
	
	\newpage
	
	\section{Proponents}
	
	Scott Morgan \hspace{3mm} $\vert$ \hspace{3mm} Cardiff University \\
	Claire George \hspace{3mm} $\vert$ \hspace{3mm} Bridgend College
	
	\section{Title} 
	
	Raspberry Pi Robo Rally 
	
	\section{Proposal Statement}
	
	To design, implement and test an application capable of controlling a Raspberry Pi CamJam Robot from a mobile smartphone.
	
	\section{Objectives}

	\begin{itemize}
		\item Discuss ways of controlling the Pi remotely and decide on primary course of action.  
		\item Design and wireframe an application.
		\item Implement application in code, using software and examples provided by the instructor.
		\item Export application to an emulator or real device.
		\item Iteratively test design.
		\item Gather user feedback and update as necessary.
	\end{itemize}
	
	\section{Potential Audience}
	
	The app must be user-friendly and accessible to a wide range of ages and technological abilities, from primary school children through to adults. 
	
	\section{Recommended Hardware \& Software}
	
	\begin{itemize}
		\item Raspberry Pi 3 or Zero W
		\item CamJam Edukit 3
		\item Python Idle 3
		\item MIT App Inventor
	\end{itemize}
	
	\section{Required Timeframe}
	
	To be completed by the Technology Skills Workshops 15-19 January 2018.
	
	\section{Required Documentation}
	
	\begin{itemize}
		\item Evidence of app design and iterative process arriving at final product. 
		\item Evidence of testing procedure. 
		\item Evidence of feedback from project managers. This could be in the form of questionnaires or live feedback after a presentation to an audience.
		\item Evidence of updates must be provided. 
		\item Justification of which feedback items were implemented in the final design. Evidence of justification can be provied in a report format or a presentation to an audience. 
	\end{itemize}

	\section{Things to Consider}
	
	\begin{enumerate}
		\item \textbf{Keep it simple!} 
		\begin{itemize}
			\item Your app does not have to be complicated. 
			\item You would be better off with a simple app which works well, than a complicated app that no one can use.
		\end{itemize}
		\item Keep a record of \textbf{EVERYTHING} you do. It'll all help towards providing evidence for your assessment.
		\item Keep in touch - if you feel like you're falling behind, let me know and we'll sort it out!
		\item See 1.  
		\begin{itemize}
			\item \textit{Simple can be harder than complex; you have to work hard to get your thinking clean to make it simple. (Steve Jobs)}
		\end{itemize}
	\end{enumerate}
	
\end{document}